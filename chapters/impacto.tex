\chapter{Social Impact}
\label{imp:impacto}

Being able to imagine \gls{ai}'s monstrous magnitude of power, can we assert whether or not it will do good to society? There is a lot of controversy around this question and arguments can be found on both sides.

Throughout history it has been clear that mankind always looks for tools to finish tasks faster, easier, more effectively, and more conveniently. Thus, \gls{ai} has also managed to find and stay in a spot for the last decade~\cite{tai2020impact}.

There is a possible dark side to what \gls{ai} may provide humanity economically speaking. There may come a time when human labor will no longer be needed as everything can be done mechanically. Humanity will need to adapt, for example by taking care of inherently human tasks, i.e, tasks related with empathy, emotions or those that are socially oriented. Nevertheless, some people like Elon Musk, with a more optimistic approach to the growth of \gls{ai}, believe there is a pretty good chance we end up with a universal basic income, due to automation~\cite{elonBasicIncome}. In contrast, there are other more pessimistic figurehead that envision a dystopic future lead by machine minds rather than human hearts.

On the other side, there are several areas where \gls{ai} can have a predictable positive impact such as healthcare, where it can provide multiple benefits: a) Faster and more accurate diagnostics ;b) Reduce errors related to human fatigue ; c) Improved radiology; d) Socially therapeutic robots.

Hence, regarding the social impact on financial markets, \gls{ai} can establish a thriving development framework for economic growth given its ease at recognizing patterns. Nonetheless, this can also become a drawback since inequality could be created as investors can end up taking the major share of the earnings, widening the gap between the rich and the poor.