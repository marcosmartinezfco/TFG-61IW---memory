\chapter{Conclusions}

\section{Social Impact}
\label{imp:impacto}

Being able to imagine \gls{ai}'s monstrous magnitude of power, can we assert whether or not it will do good to society? There is a lot of controversy around this question and arguments can be found on both sides.

Throughout history it has been clear that mankind always looks for tools to finish tasks faster, easier, more effectively, and more conveniently. Thus, \gls{ai} has also managed to find and stay in a spot for the last decade~\cite{tai2020impact}.

There is a possible dark side to what \gls{ai} may provide humanity economically speaking. There may come a time when human labor will no longer be needed as everything can be done mechanically. Humanity will need to adapt, for example by taking care of inherently human tasks, i.e, tasks related with empathy, emotions or those that are socially oriented. Nevertheless, some people like Elon Musk, with a more optimistic approach to the growth of \gls{ai}, believe there is a pretty good chance we end up with a universal basic income, due to automation~\cite{elonBasicIncome}. In contrast, there are other more pessimistic figurehead that envision a dystopic future lead by machine minds rather than human hearts.

On the other side, there are several areas where \gls{ai} can have a predictable positive impact such as healthcare, where it can provide multiple benefits: a) Faster and more accurate diagnostics ;b) Reduce errors related to human fatigue ; c) Improved radiology; d) Socially therapeutic robots.

Hence, regarding the social impact on financial markets, \gls{ai} can establish a thriving development framework for economic growth given its ease at recognizing patterns. Nonetheless, this can also become a drawback since inequality could be created as investors can end up taking the major share of the earnings, widening the gap between the rich and the poor.

\section{Future Lines of Work}
\label{fu:future}

During the development phase of the thesis some aspects were discussed in a brief manner, since providing a deeper insight would be the subject of a whole thesis, thus falling outside the scope proposed for this paper.

One of the facets that could be expanded is the micro-service architecture, which we talked about in section \ref{Micro-services Architecture}. We explained the main advantages this pattern could offer in terms of scalability and throughput. One line of work would be to focus on concurrency and performance scaling.

Similarly, orchestration is another key point that could be further developed. For simplicity reasons, Docker compose has been used which means that all containers run on the same host machine. One line of work would be to explore other orchestration tools and techniques such as Kubernetes, and make use of cloud computing and auto-scaling processes to maximize availability and scalability.

Finally, the prediction module is one, if not the most, important component since it is responsible of providing the predictions. Throughout the entire thesis we have compared multiple approaches and encountered multiple difficulties. One line of work would be to develop more complex models putting special attention on the training data. More parameters could be included as part of the information to feed the models, since the stock market is influenced by multiple factors as pointed in section \ref{ch:introduccion}.