\chapter{Future lines of work}
\label{fu:future}

During the development phase of the thesis some aspects were discussed in a brief manner, since providing a deeper insight would be the subject of a whole thesis, thus falling outside the scope proposed for this paper.

One of the facets that could be expanded is the micro-service architecture, which we talked about in section \ref{Micro-services Architecture}. We explained the main advantages this pattern could offer in terms of scalability and throughput. One line of work would be to focus on concurrency and performance scaling.

Similarly, orchestration is another key point that could be further developed. For simplicity reasons, Docker compose has been used which means that all containers run on the same host machine. One line of work would be to explore other orchestration tools and techniques such as Kubernetes, and make use of cloud computing and auto-scaling processes to maximize availability and scalability.

Finally, the prediction module is one, if not the most, important component since it is responsible of providing the predictions. Throughout the entire thesis we have compared multiple approaches and encountered multiple difficulties. One line of work would be to develop more complex models putting special attention on the training data. More parameters could be included as part of the information to feed the models, since the stock market is influenced by multiple factors as pointed in section \ref{ch:introduccion}. 

